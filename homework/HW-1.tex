\documentclass[11pt,noanswers,addpoints]{exam}
%\documentclass[11pt,letter]{article}
\usepackage{pslatex}
\usepackage{helvet}
\renewcommand*\familydefault{\sfdefault} % Use the sans serif version of the font

\usepackage{amsmath}
\usepackage{amssymb}
\usepackage[margin=0.5in,bottom=1in]{geometry}

\usepackage{tikz}
\usepackage{array,booktabs}

\makeatletter
\let\div\@undefined                        % undefine \div
\makeatother
\DeclareMathOperator{\div}{div}
\DeclareMathOperator{\tr}{tr}
\DeclareMathOperator*{\argmax}{arg\,max}
\DeclareMathOperator*{\argmin}{arg\,min}
\DeclareMathOperator{\diag}{diag}
\DeclareMathOperator{\dist}{dist}
\DeclareMathOperator{\var}{var}
\DeclareMathOperator{\cov}{cov}
\DeclareMathOperator{\sgn}{sgn}
\DeclareMathOperator{\const}{Const.}
%\DeclareMathOperator{\Re}{Re}
%\DeclareMathOperator{\Im}{Im}
\renewcommand{\boldsymbol}[1]{\pmb{#1}}
\newcommand{\E}{\mathbb E}
\newcommand{\R}{\mathbb R}
\newcommand{\X}{\mathbf X}
\newcommand{\x}{\mathbf x}
\newcommand{\N}{\mathcal N}

\setlength\parindent{0pt}
\begin{document}

{\Large{\textbf{Machine Learning}}} \\[2mm]
\textbf{\Huge{Homework  1}}\\[2mm]

\textbf{Not collected, not graded}.


\section{Probabilities}
\begin{questions}
\question \textbf{Fruit basket}: Suppose that we have three colored boxes $r$ (red), $b$ (blue), and $g$ (green). The respective box contents and the probability $p$ of choosing a certain box are:

{\centering
\begin{tabular}{*{4}{>{\arraybackslash\raggedleft}p{0.5in}}}
\toprule
& $r$ & $g$ & $b$\tabularnewline
\cmidrule(r){2-2}\cmidrule(r){3-3}\cmidrule{4-4}
apples & 3 & 1 & 3\tabularnewline
oranges & 4 & 1 & 3\tabularnewline
bananas & 3 & 0 & 4\tabularnewline
\midrule
$p(\text{box})$ & 0.2 & 0.2 & 0.6\tabularnewline
\bottomrule
\end{tabular}\par}
A box is selected at random (according to $p(\text{box})$), and a piece of fruit is picked from the box (with equal probability of selecting any of the items in the box), then
\begin{parts}
\part what is the probability of selecting an apple?
\part if we observe that the selected fruit is an orange, what is the probability it came from the green box?
\part is banana a fruit in the first place? (haha....)
\end{parts}

\question \textbf{Variance} 
\begin{parts}
\part The variance of a function $f\colon\R\to\R$ is defined as $\var[f]:= \E\left[\left(f(x)-\E[f(x)]\right)^2\right]$.\\ Show that $\var[f]= \E\left[f(x)^2\right]-\E[f(x)]^2$.
\part The \textbf{covariance between two random variables} is $\cov[x,y]:=\E_{x,y}\left[(x-\E[x])(y-\E[y])\right]$.\\ Show that if two random variables $x$ and $y$ are \emph{independent}, then their covariance is zero.
\end{parts}
\end{questions}

\section{Gaussians}
\begin{questions}
\question \textbf{Marginal and posterior Gaussians}: Read and study \S2.3.2 and \S2.3.3.
%\question \textbf{Maximum entropy}: The entropy of a distribution $p(\x)$ is given by $$H[p]:=-\int p(\x)\ln p(\x)\ d\x.$$
%We wish to maximize $H[p]$ over all distributions $p(\x)$ subject to the constraints that $p(\x)$ be normalized and that it have a zero mean and specific covariance:
%$$\begin{aligned}
%\int p(\x)\ d\x &= 1\\
%\int \x p(\x)\  d\x &= 0\\
%\int \x\x^T p(\x)\ d\x &= \Sigma
%\end{aligned}$$
%Perform variational maximization and use Lagrange multipliers addressing the constraints, to show that the maximum entropy distribution is given by the Gaussian.
\question \textbf{Maximum likelihood estimation}: Given a set of observations $\X = \{\x_1,\ldots,\x_N\}^T$, $\x_n\in\R^D$, assumed to be drawn independently from a multivariate Gaussian distribution, we can estimate the parameters of the distribution by maximum likelihood.
\begin{parts}
\part Write the corresponding log-likelihood function $\ln p(\X\mid\mu,\Sigma)$
\part Derive the MLE of the mean, $\mu_\text{ML}$
\part For the special case $D=1$, derive the MLE of the covariance, $\sigma^2_\text{ML}$
\end{parts}
\question \textbf{Bayesian inference}: Consider a $D$-dimensional Gaussian random variable $x$ with distribution $\N(x\mid \mu,\Sigma)$, where $\Sigma\in\R^{D\times D}$ is known, and for which we want to infer $\mu\in\R^D$ from a set of observations $\X = \{\x_1,\ldots,\x_N\}^T$, $\x_n\in\R^D$. Given a prior distribution $p(\mu)=\N(\mu\mid\mu_0,\Sigma_0)$, find the corresponding posterior distribution $p(\mu|\X)$.
\end{questions}

\end{document}