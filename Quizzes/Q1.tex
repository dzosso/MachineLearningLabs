\documentclass[11pt,noanswers,addpoints]{exam}
%\documentclass[11pt,letter]{article}
\usepackage{pslatex}
\usepackage{helvet}
\renewcommand*\familydefault{\sfdefault} % Use the sans serif version of the font

\usepackage{amsmath}
\usepackage{amssymb}
\usepackage[margin=0.5in,bottom=1in]{geometry}

%\usepackage{fullpage}
\usepackage{amsmath}
\usepackage{amssymb}
\usepackage{tikz}
\usepackage{array,booktabs}

\makeatletter
\let\div\@undefined                        % undefine \div
\makeatother
\DeclareMathOperator{\div}{div}
\DeclareMathOperator{\tr}{tr}
\DeclareMathOperator*{\argmax}{arg\,max}
\DeclareMathOperator*{\argmin}{arg\,min}
\DeclareMathOperator{\diag}{diag}
\DeclareMathOperator{\dist}{dist}
\DeclareMathOperator{\var}{var}
\DeclareMathOperator{\cov}{cov}
\DeclareMathOperator{\sgn}{sgn}
\DeclareMathOperator{\const}{Const.}
%\DeclareMathOperator{\Re}{Re}
%\DeclareMathOperator{\Im}{Im}
\renewcommand{\boldsymbol}[1]{\pmb{#1}}
\newcommand{\E}{\mathbb E}
\newcommand{\R}{\mathbb R}
\newcommand{\X}{\mathbf X}
\newcommand{\x}{\mathbf x}
\newcommand{\N}{\mathcal N}

\setlength\parindent{0pt}
\begin{document}
{\Large{\textbf{Machine Learning}}} \\[2mm]
\textbf{\Huge{Quiz 1}}\\


\hfill\hfill\makebox[0.5\textwidth]{Student Name:\enspace\hrulefill}


\begin{questions}
\question[1] Let $X\in\R$ be a continuous random variable with realizations $x$, and let the variable be distributed according to $p(x)$. Give two necessary conditions on $p(x)$ for it to be a proper probability density function (pdf):\vspace{1.5cm}\bigskip
\question[1]  Let $Y$ be another continuous random variable with realizations $y$, and be $p(x,y)$ their joint pdf. How does one obtain the marginal $p(x)$ from the joint $p(x,y)$? $$p(x) = \qquad\qquad\qquad\qquad\qquad\qquad\qquad$$\bigskip
\question[1] Let $p(x\mid y)$ be the conditional pdf of `$x$ given $y$'. How does it relate to the joint and marginal distributions? $$p(x\mid y) = \qquad\qquad\qquad\qquad\qquad\qquad\qquad$$\bigskip
\question[1] How does the conditional pdf $p(x\mid y)$ relate to its counterpart $p(y\mid x)$?$$ p(x\mid y) =\qquad\qquad\qquad\qquad\qquad\qquad\qquad$$

\end{questions}


\end{document}