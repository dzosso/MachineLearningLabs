\documentclass[11pt,noanswers,addpoints]{exam}
%\documentclass[11pt,letter]{article}
\usepackage{pslatex}
\usepackage{helvet}
\renewcommand*\familydefault{\sfdefault} % Use the sans serif version of the font

\usepackage{amsmath}
\usepackage{amssymb}
\usepackage[margin=0.5in,bottom=1in]{geometry}

%\usepackage{fullpage}
\usepackage{amsmath}
\usepackage{amssymb}
\usepackage{tikz}
\usepackage{array,booktabs}

\makeatletter
\let\div\@undefined                        % undefine \div
\makeatother
\DeclareMathOperator{\div}{div}
\DeclareMathOperator{\tr}{tr}
\DeclareMathOperator*{\argmax}{arg\,max}
\DeclareMathOperator*{\argmin}{arg\,min}
\DeclareMathOperator{\diag}{diag}
\DeclareMathOperator{\dist}{dist}
\DeclareMathOperator{\var}{var}
\DeclareMathOperator{\cov}{cov}
\DeclareMathOperator{\sgn}{sgn}
\DeclareMathOperator{\const}{Const.}
%\DeclareMathOperator{\Re}{Re}
%\DeclareMathOperator{\Im}{Im}
\renewcommand{\boldsymbol}[1]{\pmb{#1}}
\newcommand{\E}{\mathbb E}
\newcommand{\R}{\mathbb R}
\newcommand{\X}{\mathbf X}
\newcommand{\A}{\mathbf A}
\newcommand{\x}{\mathbf x}
\newcommand{\w}{\mathbf w}
\renewcommand{\t}{\mathbf t}
\newcommand{\y}{\mathbf y}
\renewcommand{\b}{\mathbf b}
\newcommand{\N}{\mathcal N}

\setlength\parindent{0pt}
\begin{document}
{\Large{\textbf{Machine Learning}}} \\[2mm]
\textbf{\Huge{Quiz 4}}

\hfill\hfill\makebox[0.5\textwidth]{Student Name:\enspace\hrulefill}\\


Consider the linear regression model $t = w_0 + \w^T\x + \epsilon$, where $\x\in\R^D$ and $\epsilon$ is Gaussian noise with zero-mean and known covariance $\sigma^2\mathcal I$. We are given $N$ samples $\X=\begin{bmatrix}\x_1&\ldots&\x_N\end{bmatrix}^\top$ with corresponding target variable $\t=\begin{bmatrix}t_1 & \ldots & t_N\end{bmatrix}^\top$.
\begin{questions}
\question[1] What and how many parameters are to be learned from the data? How many degrees of freedom?\vspace{1in}
\question[1] State the maximum likelihood estimation problem for the parameters. (abstractly) \vspace{1.5in}
\question[2] Take the negative logarithm, then state the corresponding error function/loss minimization. (specific)
\end{questions}


\end{document}